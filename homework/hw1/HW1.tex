\documentclass[12pt,letterpaper]{hmcpset}
\usepackage[margin=1in]{geometry}
\usepackage{graphicx}
\usepackage{amsthm}
\usepackage{enumitem}

\input{macros.tex}

% info for header block in upper right hand corner
\name{}
\class{Math178 SU19}
\assignment{Homework 1}
\duedate{Due: Wed, May  20, 2020}

\renewcommand{\labelenumi}{{(\alph{enumi})}}


\begin{document}
Feel free to work with other students, but make sure you write up the homework
and code on your own (no copying homework \textit{or} code; no pair programming).
Feel free to ask students or instructors for help debugging code or whatever else,
though.\\

\textit{Note:} You need to create a Github account for submission of the coding part of the homework. Please create a repository on Github to hold all your code and include your Github account username as part of the answer to the coding problems.

\begin{problem}[1]
(\textbf{Covariance. (Lecture 1 page 17)}) The covariance between two random variables $X$ and $Y$ is defined as:
$$
   \cov[X,Y] = \EE[(X-\EE[X])(Y-\EE[Y])].
$$
Prove that
$$
\cov[X,Y] = \EE[XY]- \EE[X] \EE[Y].
$$
\end{problem}
\begin{solution}
    \vfill
\end{solution}
\newpage




\begin{problem}[2]
(\textbf{Correlation. (Lecture 1 page 18)}) The correlation between two random variables $X$ and $Y$ is defined as:
$$
\text{corr}[X,Y] =\frac{ \cov[X,Y]}{\sqrt{\var[X]\var[Y]}}.
$$
Prove that
\begin{enumerate}
\item $-1\le\text{corr}[X,Y]\le1$;
\item $\text{corr}[X,Y]=1$ if and only if $Y=aX+b$ for some parameters $a\neq 0 $ and $b$.
\end{enumerate}


\end{problem}
\begin{solution}
    \vfill
\end{solution}
\newpage

\begin{problem}[3]
	(\textbf{Parametrization. (Lecture 1 page 50)}) Let $\alpha(t)$ be a parametrized curve which does not pass through the origin. If $\alpha(t_0)$ is the point of the trace of $\alpha$ closest to the origin and $\alpha'(t_0)\neq0,$ show that the position vector $\alpha(t_0)$ is orthogonal to $\alpha'(t_0)$.


\end{problem}
\begin{solution}
	\vfill
\end{solution}
\newpage




\begin{problem}[4]
	(\textbf{Extra credit. (Lecture 1 page 52)}) How to create a transformation from the data on some helix to the data of the instructor’s trajectory?


\end{problem}
\begin{solution}
	\vfill
\end{solution}
\newpage


\begin{problem}[5]
	(\textbf{Coding. (Lecture 1 page 54-70)}) Please download the H-MOG dataset from: http://www.cs.wm.edu/~qyang/hmog.html (see also Lecture 1 page 54).
	 Please read through the data description and do some visualizations if you have time. If you have any visualization result, please email or print out to submit.


\end{problem}
\begin{solution}
	\vfill
\end{solution}
\newpage
\end{document}

